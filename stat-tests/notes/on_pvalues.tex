\documentclass[12pt]{article}
\usepackage{amsmath}
\usepackage{graphicx}
\usepackage{hyperref}
\usepackage[latin1]{inputenc}

\title{Getting started}
\author{Veloci Raptor}
\date{03/14/15}

\begin{document}
PDG: https://pdg.lbl.gov/2018/reviews/rpp2018-rev-statistics.pdf
\\\\
The p-value (we'll call it p) is defined as follows:
\begin{align}
p(t_{obs}) & = \int_{t_{obs}}^{\infty}f(t|H_0)dt \equiv Pr(t \geq t_{obs}) \,\,\, \text{(See PDG, eq.39.45)} \\
& \equiv 1 - Pr(t \leq t_{obs}) \equiv 1 - \int_{-\infty}^{t_{obs}}f(t|H_0)dt \\
& \equiv 1 - \Phi(t_{obs})
\end{align}
where f is the p.d.f. of the sample is determined by the hypothesis $H_0$ and where $\Phi$ is defined as the cumulative distribution function (CDF).
\\\\
With the inverse of eq.3 above:
\begin{equation}
t_{obs} = \Phi^{-1}(1-p(t_{obs}))
\end{equation}
So what the PDG is saying in eq.39.46 is simply the special definition:
\begin{equation}
t_{obs} = 0 + Z*1 = Z = \Phi_n^{-1}(1-p(t_{obs}))
\end{equation}
where regardless of what $\Phi$ is, the Z significance is defined with $\Phi_n$ to be and I quote ``the cumulative distribution of the standard Gaussian'' so that $f_n=\mathcal{G}(\mu=0,\sigma=1)$.
\\\\
Now, why in our case I'm saying that $Z'=(\mu_{H1} -  \mu_{H0})/\sigma_{H0} = \sqrt{\chi^2}$ is equivalent to $\Phi_n^{-1}(1-p(t_{obs})) $? 

\\\\
Because:
\begin{itemize}
\item I'm assuming $f(t|H_0)$ to be gaussian with $\mu_{H0} = Obs(\text{central-replica})$ and $\sigma_{H0} = (\sigma^2_{sys}+\sigma^2_{stat})^{0.5}$. hence I'll call $\Phi \equiv \Phi_g$
\item for $t_{obs} = \mu_{H1} \equiv \mu_{H0} + Z'\sigma_{H0}$ we get:
\begin{align}
t_{obs} &= \Phi_g^{-1}(1-p(t_{obs}))\\
\mu_{H0} + Z'\sigma_{H0} & = \mu_{H0} + \sigma_{H0}\Phi_n^{-1}(1-p(t_{obs}))\\
 Z' \equiv Z &= \Phi_n^{-1}(1-p(t_{obs}))
\end{align}
\end{itemize}


\end{document}
